% -----------------------------------------------------------------
% QBIND whitepaper v2 – executive summary
% Page 1 – high-level overview for investors and protocol reviewers
% Content must stay consistent with current code and design docs.
% -----------------------------------------------------------------

\section{Executive Summary}

QBIND is a post-quantum Layer-1 blockchain designed to provide cryptographic security against adversaries with access to large-scale quantum computers. Unlike retrofit approaches that layer post-quantum primitives onto classical protocols, QBIND is purpose-built from the ground up with quantum-resistant cryptography as a hard requirement across all consensus-critical operations.

\subsection{Motivation and Problem Statement}

Classical blockchain systems rely on cryptographic assumptions---Elliptic Curve Discrete Logarithms (ECDSA, Ed25519), integer factorization (RSA), and Diffie-Hellman key exchange---that are vulnerable to Shor's algorithm running on fault-tolerant quantum computers. The timeline for such quantum capability is uncertain, with expert estimates ranging from ten to thirty years, but the \textbf{harvest-now-decrypt-later} threat model makes migration urgent: adversaries archiving signed transactions and encrypted network traffic today can forge signatures and decrypt secrets once quantum capability is achieved.

For blockchain assets intended to persist for decades---institutional custody, long-horizon reserve holdings, and cryptographic audit trails---waiting until quantum computers become practical is not an acceptable security posture. Hybrid bolt-on solutions face fundamental limitations: if the classical layer is compromised, the security model degrades entirely. A pure post-quantum Layer-1, with no classical fallback in the consensus path, provides a cleaner security argument and a more auditable trust boundary.

\subsection{High-Level Design and Architecture}

QBIND is built on five architectural pillars:

\begin{itemize}
    \item \textbf{Pure PQC-only cryptographic stack.} All consensus-critical operations---validator signatures, transaction authorization, batch certificates, and governance approvals---use NIST-standardized post-quantum primitives. No classical signature schemes or key exchange mechanisms appear in the Layer-1 consensus path.

    \item \textbf{DAG-based mempool for data availability.} Validators form transaction batches, exchange signed acknowledgments, and produce batch certificates once a $2f+1$ quorum has attested to data availability. This separation of data dissemination from ordering enables high-throughput parallel propagation while preventing data-withholding attacks.

    \item \textbf{HotStuff-style BFT consensus.} A deterministic, leader-based consensus protocol provides predictable finality through a three-chain commit rule. Safety is guaranteed even under Byzantine faults affecting up to one-third of validators.

    \item \textbf{KEMTLS-style secure networking.} All validator-to-validator communication is authenticated and encrypted using post-quantum key encapsulation. Session keys are derived via HKDF from encapsulated secrets, providing forward secrecy against future quantum adversaries.

    \item \textbf{Security-first engineering.} The protocol includes explicit test harnesses for chaos conditions, adversarial fee market behavior, and multi-region latency profiles. Benchmarking and soak testing validate deterministic execution across sustained workloads.
\end{itemize}

\subsection{Security and Cryptography}

QBIND uses \textbf{ML-DSA-44} (FIPS 204) for all digital signatures and \textbf{ML-KEM-768} (FIPS 203) for key encapsulation. These algorithms are NIST-backed post-quantum primitives that have undergone extensive cryptanalytic review during the NIST PQC standardization process. The protocol makes no classical assumptions in the Layer-1 consensus path: there is no ECDSA, no Ed25519, no RSA, and no Diffie-Hellman.

Cryptographic agility is a first-class design principle. The protocol includes versioned suite identifiers and domain separation tags, enabling future governed rotation to alternative post-quantum schemes if cryptanalysis or regulatory requirements demand it. Hardware security module (HSM) integration and operational security practices---including key rotation, remote signing, and documented recovery procedures---are supported as part of institutional-grade validator operations.

\subsection{Economics, Incentives, and Governance}

Post-quantum cryptography imposes higher computational, bandwidth, and storage costs compared to classical schemes: ML-DSA-44 signatures are approximately 38 times larger than ECDSA signatures, and verification is correspondingly more expensive. The monetary engine is explicitly designed to compensate validators for these costs through a security-budget-driven inflation model that transitions from inflation-funded to fee-funded security as network adoption matures.

The fee market is analyzed under adversarial conditions---spam attacks, front-running patterns, and mempool churn---with per-sender quotas and eviction rate limiting ensuring that honest participants maintain meaningful inclusion rates. Fee distribution follows a hybrid model combining token burn with proposer rewards, providing both deflationary pressure and direct validator incentives.

Governance follows an off-chain council model for the initial release, with a Protocol Council signing upgrade envelopes using post-quantum signatures. Upgrade envelopes specify protocol versions, activation heights, and binary hashes, providing cryptographic accountability for all consensus-affecting changes. The governance model is designed to support future migration to on-chain voting as the protocol matures.

\subsection{Roadmap and Scope of the Whitepaper}

This executive summary provides a high-level introduction to QBIND for investors, auditors, and protocol engineers. The subsequent sections of this whitepaper describe the protocol in technical depth:

\begin{itemize}
    \item \textbf{Protocol architecture}: Detailed specification of execution, the DAG mempool, HotStuff consensus, and KEMTLS networking.
    \item \textbf{Cryptographic and security analysis}: Algorithm choices, threat models, key management, and HSM integration.
    \item \textbf{Monetary policy and token economy}: Security budget formulas, inflation phases, fee distribution, and governance-tunable parameters.
    \item \textbf{Governance and upgrades}: Upgrade classes, council procedures, and the roadmap to on-chain governance.
    \item \textbf{Benchmarking and deployment}: Performance targets, operational runbooks, and launch readiness criteria.
\end{itemize}

The goal is to provide sufficient detail for independent security review, protocol implementation, and informed investment decisions.

% TODO: add a high-level system overview figure in the architecture section (not on this page).