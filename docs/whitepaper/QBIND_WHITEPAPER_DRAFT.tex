\documentclass[11pt,a4paper]{article}

% Packages
\usepackage[utf8]{inputenc}
\usepackage[T1]{fontenc}
\usepackage{lmodern}
\usepackage[margin=1in]{geometry}
\usepackage{graphicx}
\usepackage{xcolor}
\usepackage{hyperref}
\usepackage{amsmath}
\usepackage{amssymb}
\usepackage{tikz}
\usetikzlibrary{arrows.meta,positioning,shapes.geometric,calc}
\usepackage{listings}
\usepackage{fancyhdr}
\usepackage{titlesec}

% Color scheme
\definecolor{qbindblue}{RGB}{0,102,204}
\definecolor{qbindgray}{RGB}{64,64,64}
\definecolor{codegreen}{RGB}{0,128,0}
\definecolor{codegray}{RGB}{128,128,128}
\definecolor{codepurple}{RGB}{128,0,128}

% Hyperref setup
\hypersetup{
    colorlinks=true,
    linkcolor=qbindblue,
    filecolor=qbindblue,
    urlcolor=qbindblue,
    citecolor=qbindblue,
    pdftitle={QBIND: A Post-Quantum Layer-1 for Long-Horizon Security},
    pdfauthor={QBIND Protocol Team},
}

% Code listing style
\lstdefinestyle{qbindcode}{
    backgroundcolor=\color{gray!10},
    commentstyle=\color{codegreen},
    keywordstyle=\color{qbindblue}\bfseries,
    numberstyle=\tiny\color{codegray},
    stringstyle=\color{codepurple},
    basicstyle=\ttfamily\small,
    breakatwhitespace=false,
    breaklines=true,
    captionpos=b,
    keepspaces=true,
    numbers=left,
    numbersep=5pt,
    showspaces=false,
    showstringspaces=false,
    showtabs=false,
    tabsize=2,
    frame=single,
    rulecolor=\color{gray!30}
}
\lstset{style=qbindcode}

% Header and footer
\pagestyle{fancy}
\fancyhf{}
\fancyhead[L]{\textcolor{qbindgray}{\textit{QBIND Whitepaper}}}
\fancyhead[R]{\textcolor{qbindgray}{\thepage}}
\renewcommand{\headrulewidth}{0.4pt}

% Title formatting
\titleformat{\section}
  {\color{qbindblue}\Large\bfseries}
  {\thesection}{1em}{}

\titleformat{\subsection}
  {\color{qbindblue}\large\bfseries}
  {\thesubsection}{1em}{}

% Document metadata
\title{
    \vspace{-2cm}
    \Huge\textbf{\textcolor{qbindblue}{QBIND}}\\[0.5em]
    \Large A Post-Quantum Layer-1 for Long-Horizon Security\\[1em]
    \large\textit{Technical Whitepaper}\\[0.5em]
    \normalsize Version 0.1.0-draft
}
\author{QBIND Protocol Team}
\date{\today}

\begin{document}

\maketitle
\thispagestyle{empty}

\vspace{2em}

% ABSTRACT
\begin{abstract}
\noindent
QBIND is a Layer-1 blockchain designed to remain secure in a world where classical cryptography is no longer trustworthy. The protocol is built from the ground up on NIST-standardized post-quantum primitives (ML-DSA-44 signatures and ML-KEM-768 key encapsulation), coupled with a DAG-based data availability layer and HotStuff-style BFT consensus. All validator networking is secured with KEMTLS-style channels, and the economic design explicitly funds the higher cost of post-quantum cryptography while preserving long-term flexibility through on-chain governance and cryptographic agility.
\end{abstract}

\vspace{1em}
\noindent\textbf{Keywords:} Post-Quantum Cryptography, ML-DSA-44, ML-KEM-768, KEMTLS, HotStuff BFT, DAG Data Availability, Blockchain, Layer-1

\vspace{2em}

\tableofcontents
\newpage

% SECTION 1: INTRODUCTION
\section{Introduction}

\subsection{What QBIND Solves}

Classical proof-of-stake chains rely on cryptographic assumptions (ECDSA, Ed25519, pairing-based SNARKs) that are known to break under sufficiently powerful quantum adversaries. This creates a \textbf{long-horizon risk}: the historical ledger, validator keys, and rollup proofs can all become retroactively forgeable. QBIND removes this dependency at the base layer.

Every consensus-critical operation—validator signatures, batch certificates, P2P transport, governance envelopes—is built on post-quantum primitives, and the protocol is engineered so these primitives can be rotated or upgraded if future cryptanalysis demands it.

\subsection{High-Level Architecture}

At a high level, QBIND combines:

\subsubsection{Post-Quantum Cryptography as a Hard Requirement}

\begin{itemize}
    \item \textbf{ML-DSA-44 signatures} for transactions, consensus votes, batch certificates, and governance.
    \item \textbf{ML-KEM-768} for KEMTLS-style validator transport, with HKDF-derived session keys and AEAD-secured streams (ChaCha20-Poly1305).
\end{itemize}

\subsubsection{DAG-Based Data Availability + HotStuff BFT}

\begin{itemize}
    \item Validators form transaction batches, exchange ML-DSA-44-signed acknowledgements, and produce \textbf{batch certificates} once $2f+1$ validators have the data.
    \item HotStuff consensus operates on certified DAG frontiers—the set of certified batches not yet committed—ensuring that every block references only data for which a Byzantine-resistant quorum of validators have issued signed acknowledgments.
    \item This coupling prevents data-withholding attacks and ensures committed blocks are provably available.
\end{itemize}

\subsubsection{Execution and State with Explicit PQC Cost Modeling}

\begin{itemize}
    \item A \textbf{VM v0 transfer engine} plus parallel execution path, with state persisted in RocksDB and protected by deterministic test harnesses.
    \item A \textbf{monetary engine} that adjusts inflation targets for PQC's higher compute, bandwidth, and storage costs, and gradually shifts security funding from inflation to fees as the network matures.
    \item Gas accounting (T168) and fee distribution policies (T193) that burn fees while providing validator rewards.
\end{itemize}

\subsubsection{Governance and Upgrades with Cryptographic Accountability}

\begin{itemize}
    \item A \textbf{protocol council} signs upgrade envelopes with ML-DSA-44, specifying protocol versions, activation heights, and binary hashes (SHA3-256).
    \item Node operators verify these envelopes using the \texttt{qbind-envelope verify} command, which validates M-of-N council signatures, binary authenticity, and activation parameters before coordinated deployment.
    \item Clear separation between routine updates and hard-fork-class changes, with coordinated activation at specified block heights.
\end{itemize}

\begin{figure}[h]
\centering
\fbox{\parbox{0.9\textwidth}{\centering\vspace{2cm}\textit{[Architecture Overview Diagram - To be added]}\vspace{2cm}}}
\caption{QBIND High-Level Architecture: Layers and Post-Quantum Primitives}
\label{fig:architecture}
\end{figure}

\newpage

\section{[Page 2 Content - Awaiting Input]}

\textit{This section will be populated once Page 2 content is provided.}

\newpage

\appendix
\section{Implementation References}

\subsection{Codebase Structure}
\begin{itemize}
    \item \textbf{ML-DSA-44 Implementation:} \texttt{crates/qbind-crypto/src/ml
dsa44.rs}
    \item \textbf{ML-KEM-768 Implementation:} \texttt{crates/qbind-crypto/src/ml
t
d1000top8n6s0(English)\_kem768.rs}
    \item \textbf{KEMTLS Handshake:} \texttt{crates/qbind-net/tests/t135\_ml\_kem\_768\_kemtls\_tests.rs}
    \item \textbf{DAG Mempool:} \texttt{crates/qbind-node/src/dag\_mempool.rs}
    \item \textbf{HotStuff Consensus:} \texttt{crates/qbind-consensus/}
    \item \textbf{VM v0 Execution:} \texttt{crates/qbind-ledger/src/execution.rs}
    \item \textbf{Governance Envelopes:} \texttt{crates/qbind-gov/src/envelope.rs}
\end{itemize}

\subsection{Design Documents}
\begin{itemize}
    \item DAG-Consensus Coupling: \texttt{docs/mainnet/QBIND\_DAG\_CONSENSUS\_COUPLING\_DESIGN.md}
    \item Governance \& Upgrades: \texttt{docs/gov/QBIND\_GOVERNANCE\_AND\_UPGRADES\_DESIGN.md}
    \item MainNet v0 Specification: \texttt{docs/mainnet/QBIND\_MAINNET\_V0\_SPEC.md}
\end{itemize}

\end{document}